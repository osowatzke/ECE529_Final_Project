\documentclass[conference]{IEEEtran}
\usepackage{graphicx}
\usepackage{float}
\usepackage{amsmath}
\usepackage{multirow}
\usepackage{multirow,tabularx}
\usepackage{calc}
%\usepackage
%\usepackage{cite}

\newcolumntype{P}[1]{>{\centering\arraybackslash}p{#1}}

\setlength\tabcolsep{2pt}
\newlength{\colwidth}
\setlength{\colwidth}{0.12\textwidth - 2\tabcolsep}
\newlength{\colwidthB}
\setlength{\colwidthB}{0.069\textwidth - 2\tabcolsep}
\newlength{\colwidthC}
\setlength{\colwidthC}{0.076\textwidth - 2\tabcolsep}
%\renewcommand{\tabularxcolumn}[1]{m{#1}}
%\newcolumntype{L}{>{\RaggedRight\arraybackslash}X}
%\newcolumntype{C}[1]{>{\Centering\arraybackslash%
%        \hsize=#1\hsize\linewidth=\hsize}X}
        
\begin{document}

\title{Direction of Angle Estimation}

\author{\IEEEauthorblockN{Owen Sowatzke}
\IEEEauthorblockA{\textit{Electrical Engineering Department} \\
\textit{University of Arizona}\\
Tucson, USA \\
osowatzke@arizona.edu}}
\maketitle

\begin{abstract}
	Direction of arrival (DOA) estimation plays an integral role in radar, wireless communications, sonar, radio astronomy, and navigation \cite{doa_algorithms_raghu}. Specific algorithms for DOA estimation include beamforming, MVDR, MUSIC, improved MUSIC, Root Music, and ESPRIT \cite{doa_algorithms_raghu}. Following the work of Raghu and Kamari, this paper evaulates the performance of each of these direction of arrival algorithms operating on data from a uniform linear array. 
\end{abstract}

	\section{Background}
	
	Each of the direction of angle algorithms compared in this report operate on data collected with an antenna array. This report specifically examines data collected with a uniform linear array (ULA). A block diagram of a uniform linear array with $M$ antenna elements, receiving signals from $D$ sources is given by Raghu and Kamari in \cite{doa_algorithms_raghu}.
	
	\begin{figure}[H]
		\centerline{\includegraphics[width=0.5\textwidth]{uniform_linear_array.png}}
		\caption{Uniform Linear Array \cite{doa_algorithms_raghu}}
	\end{figure}
	
	As shown in the figure above, let $d$ be the distance between successive array elements, and $\beta$ be defined as $2\pi/\lambda$. Then if each of the source signals is given as $s_1(n),...,s_D(n)$, the signal received by m-th array element can be written as:
	%
	\begin{equation}
		\label{received_signal}
		\begin{split}
			x_m(n) &= s_1(n)e^{-j{\beta}d(m-1)\sin{\theta_1}} + \cdots\\
			&+ s_D(n)e^{-j{\beta}d(m-1)\sin{\theta_D}} + w_m(n)
		\end{split}
	\end{equation}
	%
	where $w_m(n)$ denotes the noise on the m-th element \cite{doa_algorithms_raghu}.
	
	Equation (\ref{received_signal}) can be rewritten in matrix form as follows:
	%
	\begin{equation}
		X = AS + W
	\end{equation}
	%
	where $X = \begin{bmatrix} x_1(n) & \cdots & x_n(n)\end{bmatrix}^T$ is the Mx1 received signal vector, $A$ is the MxD array steering matrix, $S = \begin{bmatrix} s_1(n) & \cdots & s_D(n)\end{bmatrix}^T$ is the Dx1 source signal vector, and $W = \begin{bmatrix} w_1(n) & \cdots & w_n(n)\end{bmatrix}^T$ is the Mx1 noise signal vector \cite{doa_algorithms_raghu}. The array steering matrix $A$ can be written as:
	%
	\begin{equation}
		A = \begin{bmatrix} a(\theta_1) & \cdots & a(\theta_D) \end{bmatrix}
	\end{equation}
	%
	where $a(\theta_i)$ represents the Mx1 array response for the source signal incident at angle $\theta_i$ and is defined in \cite{doa_algorithms_raghu} as:
	%	
	\begin{equation}
		\label{array_response_vector}
	 	a(\theta_i) = \begin{bmatrix} 1 & e^{-j{\beta}d\sin(\theta_i)} & \cdots & e^{-j{\beta}d(M-1)\sin(\theta_i)}\end{bmatrix}^T
	\end{equation}
	
		Each direction of arrival algorithm leverages spatial auto-correlation matrices $R_{xx}$ defined in \cite{doa_algorithms_raghu}  using the received signal vector $X$ as:
	%
	\begin{equation}
		R_{xx} = E[XX^H]
	\end{equation}
	%
	The expected value in the above equation is estimated using $K$ (a finite number) samples of the received signal vector $X$. The resulting spatial auto-correlation estimate ${R}_{xx}$ is given in \cite{doa_algorithms_raghu} as follows: 
	%
	\begin{equation}
		\label{spatial_matrix_estimate}
		\hat{R}_{xx} = \sum_{n=1}^{K}{X(n)X^H(n)}
	\end{equation}
	
	\section{Direction of Arrival Algorithms}
	
	\subsection{Beamforming}
	
		Beamforming uses the estimate of the spatial auto-correlation matrix $\hat{R}_{xx}$ defined in Equation (\ref{spatial_matrix_estimate}) to estimate the spatial spectrum $\hat{P}_{bf}(\theta)$ \cite{doa_algorithms_raghu}. The spatial spectrum estimate $\hat{P}_{bf}(\theta)$ is defined by
		%
		\begin{equation}
			\hat{P}_{bf}(\theta) = a^H(\theta)\hat{R}_{xx}a(\theta)
		\end{equation}
		%
		where $a(\theta)$ is the array response given in Equation (\ref{array_response_vector}).
		
		To estimate the direction of arrival, compute the spatial spectrum in discrete steps over a region of interest. Then, find the peaks of the spatial spectrum. The angles corresponding to the peaks are the directions of arrival for each source.
		
	\subsection{Capon/MVDR}
	
		The Capon algorithm seeks to minimize the power of received signals in all direction except for the look angle \cite{doa_algorithms_raghu}. The Capon spatial spectrum $\hat{P}_{capon}(\theta)$ is generated using  the spatial auto-correlation matrix estimated by Equation (\ref{spatial_matrix_estimate}). For each look angle $\theta$, the spatial spectrum is given by:
	%
	\begin{equation}
		\hat{P}_{capon}(\theta) = \frac{1}{a^H(\theta){\hat{R}_{xx}}^{-1}a(\theta)}
	\end{equation}
	%
	
		Similar to beamforming direction of arrival estimation, Capon direction of angle estimation requires generating the spatial spectrum in discrete steps over a region of interest. Then, this spectrum can be searched for peaks. The angle corresponding to the peaks are the directions of arrival for each source.
	
	\subsection{MUSIC}
	
		The Multiple Signal Classification (MUSIC) algorithm takes an eigen-decomposition of the spatial auto-correlation matrix to produce orthogonal signal and noise subspaces. The eigenvectors of the resulting noise subspace are used to generate a spatial spectrum which can be used to estimate the directions of arrival \cite{doa_algorithms_raghu}.
	
		To find the signal and noise subspaces, the spatial auto-correlation matrix must decomposed into its eigenvalues and eigenvectors:
	%
	\begin{equation}
		\hat{R}_{xx} = E{\Lambda}E^H
		\label{eigenvalue_decomposition_of_Rxx}
	\end{equation}
	%
	In the above equation, $\Lambda = \text{diag}\{\lambda_0,...,\lambda_{M-1}\}$ is the MxM diagonal eigenvalue matrix and $E=\{e_0,...,e_{M-1}\}$ is the MxM eigenvector matrix.
	
	Because there are $D$ sources and $M$ eigenvalues, the $N = M - D$ smallest eigenvalues should correspond to the noise subspace. Assume the eigenvalues are sorted in ascending order (i.e. $\lambda_0 < ... < \lambda_{M-1}$). Then, $\{\lambda_0,...,\lambda_{N-1}\}$ and their corresponding eigenvectors $\{e_0,...,e_{N-1}\}$ correspond to the noise subspace.
	
	The number of source signals $D$ is typically not known, but can be estimated using a variety of methods, which include AIC and MDL \cite{num_sources_est_salman}. Both of these algorithms require a flipped copy of the eigenvalues (i.e. $\lambda_0 > ... > \lambda_{M-1}$). Using the flipped eigenvalues, the number of sources estimated with AIC is given in \cite{num_sources_est_salman} as
	%
	\begin{equation}
		\begin{split}
		\hat{D} &= argmin_n\left(-2\log\left(\frac{\prod_{i=n}^{M-1}{{\lambda_i}^{\frac{1}{M-n}}}}{\frac{1}{M - n}\sum_{i=n}^{M-1}{\lambda_i}}\right)^{(M-n)K}\right. \\
		& \left. + 2n(2M - n) \vphantom{-2\log\left(\frac{\prod_{i=n}^{M-1}{{\lambda_i}^{\frac{1}{M-k}}}}{\frac{1}{M - n}\sum_{i=n}^{M-1}{\lambda_i}}\right)^{(M-n)K}} \right)
		\end{split}
		\label{AIC_num_sources}
	\end{equation}
	where $n$ is the index of each eigenvalue. Similarly, the number of sources estimated with MDL is given in \cite{num_sources_est_salman} as
	%
	\begin{equation}
		\begin{split}
		\hat{D} &= argmin_n\left(-\log\left(\frac{\prod_{i=n}^{M-1}{{\lambda_i}^{\frac{1}{M-n}}}}{\frac{1}{M - n}\sum_{i=n}^{M-1}{\lambda_i}}\right)^{(M-n)K}\right. \\
		& \left. + \frac{1}{2}n(2M - n)\log(K) \vphantom{-\log\left(\frac{\prod_{i=n}^{M-1}{{\lambda_i}^{\frac{1}{M-k}}}}{\frac{1}{M - n}\sum_{i=n}^{M-1}{\lambda_i}}\right)^{(M-n)K}} \right)
		\end{split}
		\label{MDL_num_sources}
	\end{equation}
	
		Using the estimated number of sources $\hat{D}$, the number of noise eigenvalues/eigenvectors is approximately $\hat{N} = M - \hat{D}$. Let $E_n = \{e_0,...,e_{N-1}\}$ be the MxN noise subspace. Then, the music spectrum is given in \cite{doa_algorithms_raghu} as
	%
	\begin{equation}
		\label{music_spectrum}
		\hat{P}_{music}(\theta) = \frac{1}{a^H(\theta)E_n{E_n}^Ha(\theta)}
	\end{equation}
	
		As was the case with the other direction direction of angle algorithms, the music spectrum must be generated in discrete steps of a region of interest and then searched for peaks. The angles corresponding peaks are the angles of arrival.
		
	\subsection{Improved-MUSIC}
	
		The MUSIC algorithm produces poor angle estimates when the source signals are correlated \cite{doa_algorithms_raghu}. The Improved-MUSIC algorithm overcomes this limitation. It starts by performing a conjugate reconstruction of the received signal vector X with the matrix J \cite{doa_algorithms_raghu}.
		%	
		\begin{equation}
			Y = JX^{*}
		\end{equation}
		%
		where J is defined is defined as follows:
		%
		\begin{equation}
			J = \begin{bmatrix}
				0 & 0 & \cdots & 0 & 1\\
				0 & 0 & \cdots & 1 & 0\\
				\vdots & \vdots & \ddots & \vdots & \vdots\\
				1 & 0 & \cdots & 0 & 0
			\end{bmatrix}
		\end{equation}
		
		Then, the spatial auto-correlation matrix of the received signal vector $X$ is computed using Equation (\ref{spatial_matrix_estimate}). The spatial auto-correlation matrix of the transformed received signal vector $Y$ is given in \cite{doa_algorithms_raghu} as
		%
		\begin{equation}
			\hat{R}_{yy} = \sum_{n=1}^{K}{Y(n)Y^H(n)}
		\end{equation}
		%
		A new spatial auto-correlation matrix with the same noise subspace can be generated following \cite{doa_algorithms_raghu} as follows:
		%
		\begin{equation}
			\hat{R} = \hat{R}_{xx} + \hat{R}_{yy}
		\end{equation} 
		
		Following the MUSIC algorithm with $\hat{R}$ in place of $\hat{R}_{xx}$, a spatial spectrum can be derived for the Improved-MUSIC algorithm. The spatial spectrum can then be searched for peaks. The angles corresponding to these peaks are the directions of arrival for each source.
			
	\subsection{Root-MUSIC}
		
		The Root-MUSIC algorithm involves finding the roots of a polynomial and selecting the roots which correspond to sources. In contrast to the other algorithms presented, it does not require computing and searching through the spatial spectrum for peaks.
		
		Similar to the MUSIC algorithm, the Root-MUSIC algorithm involves computing an estimate of the spatial auto-correlation matrix and then finding the eigenvalues which correspond to the noise subspace. The Root-MUSIC algorithm then evaluates where the denominator of the MUSIC spectrum given by Equation (\ref{music_spectrum}) is zero. According to \cite{root_music_ko}, the denominator of the music spectrum can be written in the following form:
		%
		\begin{equation}
			\setlength{\arraycolsep}{2pt}
			\renewcommand{\arraystretch}{0.8}
			\begin{bmatrix}z^0 & z^{-1} & \cdots & z^{-(M-1)}\end{bmatrix}\begin{bmatrix}c_{11} & c_{12} & \cdots & c_{1M}\\ c_{21} & c_{22} & \cdots & c_{2M}\\ \vdots & \vdots & \ddots & \vdots\\ c_{M1} & c_{M2} & \cdots & c_{MM}\end{bmatrix}\begin{bmatrix}z^0 \\ z^1 \\ \vdots \\ z^{M-1}\end{bmatrix}
		\end{equation}
		%
		where $z = e^{-j{\beta}\sin{\theta}}$ and each $c_{ij}$ term is an element of the matrix $C = E_{n}{E_{n}}^H$.
		
		After performing matrix multiplication, the resulting polynomial can be expressed in the following form:
		%
		\begin{equation}
			\label{root_music_summation}
			\sum_{i=1}^{M}{\sum_{j=1}^{M}{C_{ij}z^{-i+j}}}
		\end{equation}
		%
		Grouping like terms, Equation (\ref{root_music_summation}) can be further simplified to the form given in \cite{doa_algorithms_raghu}
		%
		\begin{equation}
			\label{root_music_summation_simplified}
		 	\sum_{i=-M+1}^{M-1}{c_iz^i}
		\end{equation}
		%
		where $c_i$ is the sum of the diagonal $i$ elements above the main diagonal.
		
		Equation (\ref{root_music_summation_simplified}) can be multiplied by $z^{M-1}$ to get a polynomial with only positive powers of $z$. Then, the $2(M-1)$ roots of the polynomial can be found. $M - 1$ of these roots lie inside the the unit circle. Of these roots, the $\hat{D}$ closest to the unit circle correspond to signal roots \cite{root_music_eskandari}. The direction of arrivals for each source is then given by \cite{doa_algorithms_raghu} as 
		%
		\begin{equation}
			\theta_i = -\sin^{-1}\left[\frac{\lambda}{2{\pi}d}\text{arg}(z_i)\right]
		\end{equation}
		%
		where $z_i$ is one of the roots near the unit circle.
		
	\subsection{ESPRIT}
		
		Estimation of Signal Parameters via Rotational Invariance Technique (ESPRIT) uses rotational invariance of the signal subspace to perform direction of angle estimation \cite{root_music_ko}. It starts by dividing the M-element antenna array into two identical subarrays as illustrated in \cite{doa_algorithms_raghu}.
		
		\begin{figure}[H]
			\centerline{\includegraphics[width=0.5\textwidth]{esprit_doublets.png}}
			\caption{Antenna Array Divided into Two Subarrays \cite{doa_algorithms_raghu}}
			\label{esprit_subarrays}
		\end{figure}
		
		To perform the ESPRIT algorithm, capture $K$ samples of the received signal and estimate the spatial auto-correlation matrix using equation (\ref{spatial_matrix_estimate}). Then, perform an eigenvalue decomposition of the spatial auto-correlation matrix using equation (\ref{eigenvalue_decomposition_of_Rxx}). The output of the eigenvector decomposition will be an MxM diagonal matrix $\Lambda = \text{diag}\{\lambda_0, ..., \lambda_{M-1}\}$ and an MxM eigenvector matrix $E = \begin{bmatrix}e_0 & \cdots & e_{M-1}\end{bmatrix}$. 
		
		Next, estimate the number of sources $D$ using either equation (\ref{AIC_num_sources}) or equation (\ref{MDL_num_sources}). If the eigenvalues are sorted in descending order $\lambda_0 > \lambda_1 > \cdots > \lambda_{M-1}$, then the $D$ largest eigenvalues and their respective eigenvectors ($e_0, e_1, ..., e_{D-1}$) correspond to sources. These eigenvectors can be grouped into an MxD matrix $E_s = \begin{bmatrix} e_0 & \cdots e_{D-1} \end{bmatrix}$.
		
		$E_s$ can be further split into two submatrices $E_{s1}$ and $E_{s2}$ following to the array division shown in Fig. \ref{esprit_subarrays}. Specifically, $E_{s1}$ is the first $M-1$ rows of $E_s$, and $E_{s2}$ is the last $M-1$ rows of $E_s$. Next, compute the eigenvalue decomposition of the following combination of $E_{s1}$ and $E_{s2}$:
		
		\begin{equation}
			\begin{bmatrix}
				E_{s1}^H\\[6pt]
				E_{s2}^H
			\end{bmatrix}
			\begin{bmatrix}
				E_{s1} & E_{s2}
			\end{bmatrix}
			= V{\Lambda}V^H
		\end{equation}
		
		Following \cite{doa_algorithms_raghu}, the matrix $V$ can be partitioned into four submatrices:
		
		\begin{equation*}
			V = \begin{bmatrix}
				V_{11} & V_{12}\\[6pt]
				V_{21} & V_{22}
			\end{bmatrix}
		\end{equation*}

		The rotational operator is then defined in \cite{root_music_esprit_patwari} as
		
		\begin{equation*}
			\psi = -V_{12}V_{22}^{-1}
		\end{equation*}
		
		If the eigenvalues of $\psi$ are defined as $\hat{\phi}_k$ for $k = 1,2,...,D$, then the directions of arrival are given in \cite{doa_algorithms_raghu} as:
		
		\begin{equation}
			\hat{\theta}_k = -\sin^{-1}{\left[\frac{\text{arg}(\hat{\phi}_k)}{{\beta}d}\right]}
		\end{equation}
		
		Similar to the root music algorithm, the ESPRIT algorithm directly produces directions of arrival without requiring the user to search through a spatial spectrum.
		
	\section{Simulation}
	
		The performance of each of the above direction of arrival algorithms will be evaluated in simulation. A uniform linear array model will be used to generate data samples for each direction of arrival algorithm. This model will take the number of samples ($K$), the number of antenna elements ($M$), and the antenna element spacing ($d$) as arguments. Each of the source signals will be a complex exponential at user-configurable frequencies, directions of arrival, and signal to noise ratios.
		
		After the received signal is generated, it will be input into each of the direction of arrival algorithms. The direction of arrival algorithms will then output either a spatial spectrum or an angle estimate. These outputs will be graphed on the same axes to provide a basis for comparing algorithms. Angle estimates not directly output by an algorithms will be measured by finding peaks in the spatial spectrum. A block diagram of the simulation workflow is shown in Figure \ref{sim_workflow}.
		
		\begin{figure}[H]
			\centerline{\fbox{\includegraphics[width=0.5\textwidth]{program_flow.png}}}
			\caption{Simulation Workflow}
			\label{sim_workflow}
		\end{figure}
		
		Specific system parameters that I will vary include the number of elements in the uniform linear array, the spacing of elements in the uniform linear array, and the number of correlation samples. I will also vary the input data by adjusting the signal to noise ratio, the number of sources, the angle separation between sources, and the correlation between source signals.
		
	\section{Results}
	
		The first performance comparison examines the impact of the number of antenna elements ($M$) on algorithm performance. Two cases were run using $M = 4$ and $M=27$, while holding the remaining parameters constant. The remaining parameters included the antenna element spacing $d = \lambda/2$, $\text{SNR}=30\text{dB}$, number of sources $D=3$, number of correlation samples $K=100$, source angles $\theta = \{-40^{\circ}, 10^{\circ}, 50^{\circ}\}$, and source frequencies $\omega = \{\pi/3, \pi/5, \pi/7\}$. The resulting spatial spectrums are shown in Fig. \ref{fig::spatial_spectrum_m4} and Fig. \ref{fig::spatial_spectrum_m27}, and the corresponding directions of arrival estimates are given in Table \ref{table::doa_varying_m}.
		
		\begin{figure}[H]
			\centerline{\fbox{\includegraphics[width=0.5\textwidth]{spatial_spectrum_4_antenna_elements.png}}}
			\caption{Spatial Spectrum with M=4}			
			\label{fig::spatial_spectrum_m4}
		\end{figure}
		
		\begin{figure}[H]
			\centerline{\fbox{\includegraphics[width=0.5\textwidth]{spatial_spectrum_27_antenna_elements.png}}}
			\caption{Spatial Spectrum with M=27}
			\label{fig::spatial_spectrum_m27}
		\end{figure}

		\begin{table}[H]
		\caption{DOA Algorithm Performance for M=4 and M=27}
		\footnotesize
		\begin{tabular}{|P{\colwidth}|P{\colwidth}|P{\colwidth}|P{\colwidth}|}
			\hline
			\multirow{3}{\colwidth}{\centering\textbf{DOA Algorithms}} & \multicolumn{3}{c|}{\textbf{Direction of Arrival (in degrees)}}\\
			\cline{2-4}
			& \multirow{2}{\colwidth}{\centering\textbf{Actual DOA}} & \multirow{2}{\colwidth}{\centering\textbf{Estimated DOA for M=4} }& \multirow{2}{\colwidth}{\centering\textbf{Estimated DOA for M=27}} \\
			& & & \\
			\hline
			\multirow{3}{\colwidth}{\centering Beamforming} & $-40^{\circ}$ & $-38^{\circ}$ & $-40^{\circ}$ \\
			\cline{2-4}
			& $10^{\circ}$ & $8^{\circ}$ & $10^{\circ}$\\
			\cline{2-4}
			& $50^{\circ}$ & $49^{\circ}$ & $50^{\circ}$\\
			\hline
			\multirow{3}{\colwidth}{\centering Capon} & $-40^{\circ}$ & $-40^{\circ}$ & $-40^{\circ}$ \\
			\cline{2-4}
			& $10^{\circ}$ & $10^{\circ}$ & $10^{\circ}$\\
			\cline{2-4}
			& $50^{\circ}$ & $50^{\circ}$ & $50^{\circ}$\\
			\hline
			\multirow{3}{\colwidth}{\centering MUSIC} & $-40^{\circ}$ & $-40^{\circ}$ & $-40^{\circ}$ \\
			\cline{2-4}
			& $10^{\circ}$ & $10^{\circ}$ & $10^{\circ}$\\
			\cline{2-4}
			& $50^{\circ}$ & $50^{\circ}$ & $50^{\circ}$\\
			\hline
			\multirow{3}{\colwidth}{\centering Improved-MUSIC} & $-40^{\circ}$ & $-40^{\circ}$ & $-40^{\circ}$ \\
			\cline{2-4}
			& $10^{\circ}$ & $10^{\circ}$ & $10^{\circ}$\\
			\cline{2-4}
			& $50^{\circ}$ & $50^{\circ}$ & $50^{\circ}$\\
			\hline
			\multirow{3}{\colwidth}{\centering Root-MUSIC} & $-40^{\circ}$ & $-40.0381^{\circ}$ & $-40.0013^{\circ}$ \\
			\cline{2-4}
			& $10^{\circ}$ & $10.0183^{\circ}$ & $10.0002^{\circ}$\\
			\cline{2-4}
			& $50^{\circ}$ & $50.0782^{\circ}$ & $50.0017^{\circ}$\\
			\hline
			\multirow{3}{\colwidth}{\centering ESPRIT} & $-40^{\circ}$ & $-40.0381^{\circ}$ & $-40.0039^{\circ}$ \\
			\cline{2-4}
			& $10^{\circ}$ & $10.0183^{\circ}$ & $9.99868^{\circ}$\\
			\cline{2-4}
			& $50^{\circ}$ & $50.0782^{\circ}$ & $50.0060^{\circ}$\\
			\hline
		\end{tabular}		
		\label{table::doa_varying_m}
		\end{table}
		
		Comparing Fig. \ref{fig::spatial_spectrum_m4} and Fig. \ref{fig::spatial_spectrum_m27} along with the angle estimates given in Table \ref{table::doa_varying_m}, algorithm performance degrades as the number of antenna elements is decreased. The beamforming alogorithm appears most sensitive to the number of antenna elements. The sidelobes in the beamforming spatial spectrum occur less frequently for the $M=4$ case, but are of much higher in amplitude. This can result in lower power targets being obscured in the spatial spectrum. The Capon spatial spectrum appears moderately affected by the number of antenna elements. The spatial spectrum floor is approximately $15\text{dB}$ higher for the $M=4$ case than the $M=27$ case. The performance of the remaining algorithms appear to only marginally degraded as the number of antenna elements decreases.
		
		The next performance comparison examines the effects of the number of correlation samples ($K$) on algorithm performance. Two cases were run using $K=5$ and $K=100$, while holding the remaining parameters constant. The remaining parameters included number of antenna elements $M=10$, antenna element spacing $d = \lambda/2$, $\text{SNR}=30\text{dB}$, number of sources $D=3$, source angles $\theta = \{-40^{\circ}, 10^{\circ}, 50^{\circ}\}$, and source frequencies $\omega = \{\pi/3, \pi/5, \pi/7\}$. The resulting spatial spectrums are shown in Fig. \ref{fig::spatial_spectrum_k5} and Fig. \ref{fig::spatial_spectrum_k100}, and the corresponding directions of arrival estimates are given in Table \ref{table::doa_varying_k}.
		
		\begin{figure}[H]
			\centerline{\fbox{\includegraphics[width=0.5\textwidth]{spatial_spectrum_5_samples.png}}}
			\caption{Spatial Spectrum with K=5}
			\label{fig::spatial_spectrum_k5}
		\end{figure}
		
		\begin{figure}[H]
			\centerline{\fbox{\includegraphics[width=0.5\textwidth]{spatial_spectrum_100_samples.png}}}
			\caption{Spatial Spectrum with K=100}
			\label{fig::spatial_spectrum_k100}
		\end{figure}
		
		\begin{table}[H]
		\caption{DOA Algorithm Performance for K=5 and K=100}
		\footnotesize
		\begin{tabular}{|P{\colwidth}|P{\colwidth}|P{\colwidth}|P{\colwidth}|}
			\hline
			\multirow{3}{\colwidth}{\centering\textbf{DOA Algorithms}} & \multicolumn{3}{c|}{\textbf{Direction of Arrival (in degrees)}}\\
			\cline{2-4}
			& \multirow{2}{\colwidth}{\centering\textbf{Actual DOA}} & \multirow{2}{\colwidth}{\centering\textbf{Estimated DOA for K=5} }& \multirow{2}{\colwidth}{\centering\textbf{Estimated DOA for K=100}} \\
			& & & \\
			\hline
			\multirow{3}{\colwidth}{\centering Beamforming} & $-40^{\circ}$ & $-38^{\circ}$ & $-40^{\circ}$ \\
			\cline{2-4}
			& $10^{\circ}$ & $10^{\circ}$ & $10^{\circ}$\\
			\cline{2-4}
			& $50^{\circ}$ & $49^{\circ}$ & $50^{\circ}$\\
			\hline
			\multirow{3}{\colwidth}{\centering Capon} & $-40^{\circ}$ & $-40^{\circ}$ & $-40^{\circ}$ \\
			\cline{2-4}
			& $10^{\circ}$ & $11^{\circ}$ & $10^{\circ}$\\
			\cline{2-4}
			& $50^{\circ}$ & $50^{\circ}$ & $50^{\circ}$\\
			\hline
			\multirow{3}{\colwidth}{\centering MUSIC} & $-40^{\circ}$ & $-40^{\circ}$ & $-40^{\circ}$ \\
			\cline{2-4}
			& $10^{\circ}$ & $10^{\circ}$ & $10^{\circ}$\\
			\cline{2-4}
			& $50^{\circ}$ & $50^{\circ}$ & $50^{\circ}$\\
			\hline
			\multirow{3}{\colwidth}{\centering Improved-MUSIC} & $-40^{\circ}$ & $-40^{\circ}$ & $-40^{\circ}$ \\
			\cline{2-4}
			& $10^{\circ}$ & $10^{\circ}$ & $10^{\circ}$\\
			\cline{2-4}
			& $50^{\circ}$ & $50^{\circ}$ & $50^{\circ}$\\
			\hline
			\multirow{3}{\colwidth}{\centering Root-MUSIC} & $-40^{\circ}$ & $-39.9085^{\circ}$ & $-40.0125^{\circ}$ \\
			\cline{2-4}
			& $10^{\circ}$ & $9.79253^{\circ}$ & $10.0073^{\circ}$\\
			\cline{2-4}
			& $50^{\circ}$ & $50.1250^{\circ}$ & $50.0077^{\circ}$\\
			\hline
			\multirow{3}{\colwidth}{\centering ESPRIT} & $-40^{\circ}$ & $-39.8428^{\circ}$ & $-40.0171^{\circ}$ \\
			\cline{2-4}
			& $10^{\circ}$ & $9.45141^{\circ}$ & $10.0053^{\circ}$\\
			\cline{2-4}
			& $50^{\circ}$ & $50.9027^{\circ}$ & $50.0091^{\circ}$\\
			\hline
		\end{tabular}
		\label{table::doa_varying_k}
		\end{table}
		
		The angle estimates of all algorithms degrade with few correlation samples. The capon spatial spectrum includes spurious peaks which can result in faulty angles estimates. The floor of the MUSIC spectrum also increased in response to fewer correlation samples. This increased floor, could make detection of lower power sources more challenging. Although not as severe as the MUSIC spectrum, the floor of the Improved-MUSIC spectrum also increased when the number of correlation samples was decreased.
		 		 
		The next performance comparison will be done by varying the SNR of the source signals while holding the remaining parameters constant. This will done using two tests, one which fixes the SNR of each source at $-10\text{dB}$ and the other which fixes the SNR of each source at $30 \text{dB}$. Other parameters that are held constant for each testcase include the number of antenna elements $M=10$, the antenna element spacing $d=\lambda/2$, the number of sources $D=3$, the number of correlation samples $K=100$, the source angles $\theta = \{-40^{\circ}, 10^{\circ}, 50^{\circ}\}$, and the source frequencies $\omega = \{\pi/3, \pi/5, \pi/7\}$.
		
		\begin{figure}[H]
			\centerline{\fbox{\includegraphics[width=0.5\textwidth]{spatial_spectrum_snr_minus_10db.png}}}
			\caption{Spatial Spectrum with -10dB SNR Source Signals}
			\label{fig::spatial_spectrum_snr_minus_10db}
		\end{figure}
		
		\begin{figure}[H]
			\centerline{\fbox{\includegraphics[width=0.5\textwidth]{spatial_spectrum_snr_30db.png}}}
			\caption{Spatial Spectrum with 30dB SNR Source Signals}
			\label{fig::spatial_spectrum_snr_30db}
		\end{figure}
		
		\begin{table}[H]
		\caption{DOA Algorithm Performance for SNR=-10dB and SNR=30dB}
		\footnotesize
		\begin{tabular}{|P{\colwidth}|P{\colwidth}|P{\colwidth}|P{\colwidth}|}
			\hline
			\multirow{3}{\colwidth}{\centering\textbf{DOA Algorithms}} & \multicolumn{3}{c|}{\textbf{Direction of Arrival (in degrees)}}\\
			\cline{2-4}
			& \multirow{2}{\colwidth}{\centering\textbf{Actual DOA}} & \multirow{2}{\colwidth}{\centering\textbf{Estimated DOA for SNR=-10dB} }& \multirow{2}{\colwidth}{\centering\textbf{Estimated DOA for SNR=30dB}} \\
			& & & \\
			\hline
			\multirow{3}{\colwidth}{\centering Beamforming} & $-40^{\circ}$ & $-42^{\circ}$ & $-40^{\circ}$ \\
			\cline{2-4}
			& $10^{\circ}$ & $11^{\circ}$ & $10^{\circ}$\\
			\cline{2-4}
			& $50^{\circ}$ & $50^{\circ}$ & $50^{\circ}$\\
			\hline
			\multirow{3}{\colwidth}{\centering Capon} & $-40^{\circ}$ & $-42^{\circ}$ & $-40^{\circ}$ \\
			\cline{2-4}
			& $10^{\circ}$ & $11^{\circ}$ & $10^{\circ}$\\
			\cline{2-4}
			& $50^{\circ}$ & $50^{\circ}$ & $50^{\circ}$\\
			\hline
			\multirow{3}{\colwidth}{\centering MUSIC} & $-40^{\circ}$ & $-42^{\circ}$ & $-40^{\circ}$ \\
			\cline{2-4}
			& $10^{\circ}$ & $11^{\circ}$ & $10^{\circ}$\\
			\cline{2-4}
			& $50^{\circ}$ & $50^{\circ}$ & $50^{\circ}$\\
			\hline
			\multirow{3}{\colwidth}{\centering Improved-MUSIC} & $-41^{\circ}$ & $-40^{\circ}$ & $-40^{\circ}$ \\
			\cline{2-4}
			& $10^{\circ}$ & $11^{\circ}$ & $10^{\circ}$\\
			\cline{2-4}
			& $50^{\circ}$ & $50^{\circ}$ & $50^{\circ}$\\
			\hline
			\multirow{3}{\colwidth}{\centering Root-MUSIC} & $-40^{\circ}$ & $-42.4573^{\circ}$ & $-40.0125^{\circ}$ \\
			\cline{2-4}
			& $10^{\circ}$ & $10.9781^{\circ}$ & $10.0073^{\circ}$\\
			\cline{2-4}
			& $50^{\circ}$ & $49.9928^{\circ}$ & $50.0077^{\circ}$\\
			\hline
			\multirow{3}{\colwidth}{\centering ESPRIT} & $-40^{\circ}$ & $-40.8838^{\circ}$ & $-40.0171^{\circ}$ \\
			\cline{2-4}
			& $10^{\circ}$ & $11.5068^{\circ}$ & $10.0053^{\circ}$\\
			\cline{2-4}
			& $50^{\circ}$ & $48.0052^{\circ}$ & $50.0091^{\circ}$\\
			\hline
		\end{tabular}
		\label{table::doa_varying_snr}
		\end{table}
		
		Comparing Fig. \ref{fig::spatial_spectrum_snr_minus_10db} to Fig. \ref{fig::spatial_spectrum_snr_30db}, the performance impacts of SNR can be seen. Reduced SNR leads to greater angle errors in each of the estimated directions of arrival. Reduced SNR also degrades the sharpness of the spatial spectrums peaks, reducing spatial resolution. Furthermore, reduced SNR increases the floors of the spatial spectrums making it more difficult to detect lower power source signals.
		
		Next, the performance of each direction of arrival algorithm will be compared for closely spaced sources. For this case, the simulation will be configured with number of antenna elements $M=10$, antenna element spacing $d=\lambda/2$, $\text{SNR}=30\text{dB}$, number of sources $D=3$, number of correlation samples $K=100$, source angles $\theta = \{0^{\circ}, 10^{\circ}, 15^{\circ}\}$, and source frequencies $\omega = \{\pi/3. \pi/5, \pi/7\}$.
		
		\begin{figure}[H]
			\centerline{\fbox{\includegraphics[width=0.5\textwidth]{spatial_spectrum_closely_spaced_sources.png}}}
			\caption{Spatial Spectrum for Closely Spaced Sources}
			\label{fig::closely_spaced_sources}
		\end{figure}
		
		Examining Fig. \ref{fig::closely_spaced_sources}, the mainlobes of the beamforming spatial spectrum have merged into a single large mainlobe. This can result in only one source being detected when there are in fact 3. The peaks corresponding to each source in the remaining spectrums are preserved. However, the nulls between sources are not nearly as deep as they were for sources that were spaced further apart. This can result in missed detections of lower power sources between the spectrum peaks. The MUSIC, Improved-MUSIC, Root-MUSIC spectrum, and ESPRIT direction of arrival algorithms are classified as high resolution \cite{root_music_esprit_patwari}. This means that closely spaced targets are better resolved as illustrated in Fig. \ref{fig::closely_spaced_sources}.
		
		\begin{figure}[H]
			\centerline{\fbox{\includegraphics[width=0.5\textwidth]{spatial_spectrum_correlated_sources.png}}}
			\caption{Spatial Spectrum for Correlated Sources}
		\end{figure}
		
		\begin{table}[H]
		\caption{DOA Algorithm Performance for Correlated Signals}
		\footnotesize
		\begin{tabular}{|P{\colwidthC}|P{\colwidthB}|P{\colwidthB}|P{\colwidthB}|P{\colwidthB}|P{\colwidthB}|P{\colwidthB}|}
			\hline
			\multirow{3}{\colwidthC}{\centering\textbf{Actual DOA}} & \multicolumn{6}{c|}{\textbf{DOA Algorithms}}\\
			\cline{2-7}
			& \multirow{2}{\colwidthB}{\centering\textbf{BF}} & \multirow{2}{\colwidthB}{\centering\textbf{Capon}} & \multirow{2}{\colwidthB}{\centering\textbf{MUSIC}} & \multirow{2}{\colwidthB}{\centering\textbf{I-MUSIC}} & \multirow{2}{\colwidthB}{\centering\textbf{R-MUSIC}} & \multirow{2}{\colwidthB}{\centering\textbf{ESPRIT}}\\
			& & & & & & \\
			\hline
			\multirow{2}{\colwidthC}{\centering $-41^{\circ}$ \& $\omega=\pi/3$} & \multirow{2}{\colwidthB}{\centering $-40^{\circ}$} & \multirow{2}{\colwidthB}{\centering no peak} & \multirow{2}{\colwidthB}{\centering no peak} & \multirow{2}{\colwidthB}{\centering $-40^{\circ}$} & \multirow{2}{\colwidthB}{\centering $-17.263^{\circ}$} & \multirow{2}{\colwidthB}{\centering $-28.617^{\circ}$}\\
			& & & & & & \\
			\hline
			\multirow{2}{\colwidthC}{\centering $11^{\circ}$ \& $\omega=\pi/3$} & \multirow{2}{\colwidthB}{\centering $10^{\circ}$} & \multirow{2}{\colwidthB}{\centering no peak} & \multirow{2}{\colwidthB}{\centering no peak} & \multirow{2}{\colwidthB}{\centering $10^{\circ}$} & \multirow{2}{\colwidthB}{\centering $11.588^{\circ}$} & \multirow{2}{\colwidthB}{\centering $6.921^{\circ}$}\\
			& & & & & & \\
			\hline
			\multirow{2}{\colwidthC}{\centering $50^{\circ}$ \& $\omega=\pi/7$} & \multirow{2}{\colwidthB}{\centering $50^{\circ}$} & \multirow{2}{\colwidthB}{\centering $50^{\circ}$} & \multirow{2}{\colwidthB}{\centering $50^{\circ}$} & \multirow{2}{\colwidthB}{\centering $50^{\circ}$} & \multirow{2}{\colwidthB}{\centering $50.007^{\circ}$} & \multirow{2}{\colwidthB}{\centering $50.010^{\circ}$}\\
			& & & & & & \\
			\hline
		\end{tabular}
		\end{table}
		% Following Equation (\ref{received_signal}), the received signal will be a sum of each source signals and complex gaussian noise. The noise will be unit power and each source signal will be scaled to provide a user-configured SNR. received signal will be the sum of complex gaussian noise and and Each of the received signals will be a complex exponential tuned at a user-specified frequency received at a user-specified angle.
		
		% The uniform will contain is model will be representative of a uniform linear array with $M$ elements
		
		% angles corresponding of each signal root can be found by using the
		%The sum of the terms on the main diagonal will be weighted by $z^0$, the sum of the terms directly above the main diagonal will be weighted by $z^1$, and so on. The terms in the resulting polynomial will be weighted by powers of $z$ from $-(M-1)$ to $M-1$.
		
		%$C_ij$. jrh The terms on the main diagonal with all hav
		%this is the point at which the algorithms begin to differ.The MUSIC spectrum is then given by Equation (\ref{music_spectrum}). The denominator of the spectrum can be written in the following form:
		
	%\begin{equation}
	%	A = \begin{bmatrix}
	%		1 & 1 & \cdots & 1\\
	%		e^{j{\beta}d\sin{\theta_1}} & e^{j{\beta}d\sin{\theta_2}} & \cdots & e^{j{\beta}d\sin{\theta_D}} \\
	%		\vdots & \vdots & \ddots & \vdots\\
	%		e^{j{\beta}d(M-1)\sin{\theta_1}} & e^{j{\beta}d(M-1)\sin{\theta_2}} & \cdots & e^{j{\beta}d(M-1)\sin{\theta_D}}
	%	\end{bmatrix}
	%\end{equation}
	\bibliographystyle{IEEEtran}
	\bibliography{IEEEabrv,sources}
\end{document}
\documentclass[conference]{IEEEtran}
\usepackage{graphicx}
\usepackage{float}
\usepackage{amsmath}
%\usepackage{cite}

\begin{document}

\title{Direction of Angle Estimation}

\author{\IEEEauthorblockN{Owen Sowatzke}
\IEEEauthorblockA{\textit{Electrical Engineering Department} \\
\textit{University of Arizona}\\
Tucson, USA \\
osowatzke@arizona.edu}}
\maketitle

\begin{abstract}
	Direction of arrival (DOA) estimation plays an integral role in radar, wireless communications, sonar, radio astronomy, and navigation \cite{doa_algorithms_raghu}. Specific algorithms for DOA estimation include beamforming, MVDR, MUSIC, improved MUSIC, Root Music, and ESPRIT \cite{doa_algorithms_raghu}. Following the work of Raghu and Kamari, this paper evaulates the performance of each of these direction of arrival algorithms operating on data from a uniform linear array. 
\end{abstract}

	\section{Background}
	
	Each of the direction of angle algorithms compared in this report operate on data collected with an antenna array. This report specifically examines data collected with a uniform linear array (ULA). A block diagram of a uniform linear array with $M$ antenna elements, receiving signals from $D$ sources is given by Raghu and Kamari in \cite{doa_algorithms_raghu}.
	
	\begin{figure}[H]
		\centerline{\includegraphics[width=0.5\textwidth]{uniform_linear_array.png}}
		\caption{Uniform Linear Array \cite{doa_algorithms_raghu}}
		\label{Monte Carlo Estimate}
	\end{figure}
	
	As shown in the figure above, let $d$ be the distance between successive array elements, and $\beta$ be defined as $2\pi/\lambda$. Then if each of the source signals is given as $s_1(n),...,s_D(n)$, the signal received by m-th array element can be written as:
	%
	\begin{equation}
		\label{received_signal}
		\begin{split}
			x_m(n) &= s_1(n)e^{-j{\beta}d(m-1)\sin{\theta_1}} + \cdots\\
			&+ s_D(n)e^{-j{\beta}d(m-1)\sin{\theta_D}} + w_m(n)
		\end{split}
	\end{equation}
	%
	where $w_m(n)$ denotes the noise on the m-th element \cite{doa_algorithms_raghu}.
	
	Equation (\ref{received_signal} can be rewritten in matrix form as follows:
	%
	\begin{equation}
		X = AS + W
	\end{equation}
	%
	where $X = \begin{bmatrix} x_1(n) & \cdots & x_n(n)\end{bmatrix}^T$ is the Mx1 received signal vector, $A$ is the MxD array steering matrix, $S = \begin{bmatrix} s_1(n) & \cdots & s_D(n)\end{bmatrix}^T$ is the Dx1 source signal vector, and $W = \begin{bmatrix} w_1(n) & \cdots & w_n(n)\end{bmatrix}^T$ is the Mx1 noise signal vector \cite{doa_algorithms_raghu}. The array steering matrix $A$ can be written as:
	%
	\begin{equation}
		A = \begin{bmatrix} a(\theta_1) & \cdots & a(\theta_D) \end{bmatrix}
	\end{equation}
	%
	where $a(\theta_i)$ represents the Mx1 array response for the source signal incident at angle $\theta_i$ and is defined in \cite{doa_algorithms_raghu} as:
	%	
	\begin{equation}
	 	a(\theta_i) = \begin{bmatrix} 1 & e^{j{\beta}d\sin(\theta_i)} & \cdots & e^{j{\beta}d(M-1)\sin(\theta_i)}\end{bmatrix}^T
	\end{equation}
	
		Each direction of arrival algorithm leverages spatial auto-correlation matrices $R_{xx}$ defined in \cite{doa_algorithms_raghu}  using the received signal vector $X$ as:
	%
	\begin{equation}
		R_{xx} = E[XX^H]
	\end{equation}
	%
	The expected value in the above equation is estimated using a finite number of samples of the received signal vector $X$. The resulting spatial auto-correlation estimate ${R}_{xx}$ is given in \cite{doa_algorithms_raghu} as follows: 
	%
	\begin{equation}
		\hat{R}_{xx} = \sum_{n=1}^{K}{X(n)X^H(n)}
	\end{equation}
	
	
	%\begin{equation}
	%	A = \begin{bmatrix}
	%		1 & 1 & \cdots & 1\\
	%		e^{j{\beta}d\sin{\theta_1}} & e^{j{\beta}d\sin{\theta_2}} & \cdots & e^{j{\beta}d\sin{\theta_D}} \\
	%		\vdots & \vdots & \ddots & \vdots\\
	%		e^{j{\beta}d(M-1)\sin{\theta_1}} & e^{j{\beta}d(M-1)\sin{\theta_2}} & \cdots & e^{j{\beta}d(M-1)\sin{\theta_D}}
	%	\end{bmatrix}
	%\end{equation}
	\bibliographystyle{IEEEtran}
	\bibliography{IEEEabrv,sources}
\end{document}